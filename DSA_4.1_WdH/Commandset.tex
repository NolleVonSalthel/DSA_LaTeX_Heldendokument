\makeatletter

%--------------------------------------------------------------------
%	Talentzeug
\newcounter{TalentAnzahl}
%
% Talent: \neuesTalent{Typ}
\newcommand{\neuesTalent}[1]{%
	\addtocounter{TalentAnzahl}{1}
	\IfEqCase{#1}%
	{Kampf}{\neuesTalentKampf}%
	{Koerper}{\neuesTalentKoerper}%
	{Gesellschaft}{\neuesTalentGesellschaft}%
	{Natur}{\neuesTalentNatur}%
	{Wissen}{\neuesTalentWissen}%
	{Handwerk}{\neuesTalentHandwerk}%
	{Sprachen}{\neuesTalentSprachen}
}
%
% Kampftalent
\newcounter{TalentAnzahlKampf}
\def\TalentTabelleInhaltKampf{}
% command => \neuesTalentKampf{Name}{TaW}{AT}{PA}{BE}{Steigerung}
\newcommand{\neuesTalentKampf}[6]{%
	\addtocounter{TalentAnzahlKampf}{1}
	\g@addto@macro\TalentTabelleInhaltKampf{#1 & #6 & #5 & #3 $\bullet$ #4 & #2 &\\\hline}
}
%
% Körperliches Talent
\newcounter{TalentAnzahlKoerper}
\def\TalentTabelleInhaltKoerper{}
% command => \neuesTalentKoerper{Name}{Eig.1}{Eig.2}{Eig.3}{TaW}{BE}
\newcommand{\neuesTalentKoerper}[6]{%
	\addtocounter{TalentAnzahlKoerper}{1}
	\g@addto@macro\TalentTabelleInhaltKoerper{#1 (#2 $\bullet$ #3 $\bullet$ #4) & #6 & #5 &\\\hline}
}
%
% Gesellschaftliches Talent
\newcounter{TalentAnzahlGesellschaft}
\def\TalentTabelleInhaltGesellschaft{}
% command => \neuesTalentGesellschaft{Name}{Eig.1}{Eig.2}{Eig.3}{TaW}
\newcommand{\neuesTalentGesellschaft}[5]{%
	\addtocounter{TalentAnzahlGesellschaft}{1}
	\g@addto@macro\TalentTabelleInhaltGesellschaft{#1 (#2 $\bullet$ #3 $\bullet$ #4) & #5 &\\\hline}
}
%
% Natur-Talent
\newcounter{TalentAnzahlNatur}
\def\TalentTabelleInhaltNatur{}
% command => \neuesTalentNatur{Name}{Eig.1}{Eig.2}{Eig.3}{TaW}
\newcommand{\neuesTalentNatur}[5]{%
	\addtocounter{TalentAnzahlNatur}{1}
	\g@addto@macro\TalentTabelleInhaltNatur{#1 (#2 $\bullet$ #3 $\bullet$ #4) & #5 &\\\hline}
}
%
% Wissens Talent
\newcounter{TalentAnzahlWissen}
\def\TalentTabelleInhaltWissen{}
% command => \neuesTalentWissen{Name}{Eig.1}{Eig.2}{Eig.3}{TaW}
\newcommand{\neuesTalentWissen}[5]{%
	\addtocounter{TalentAnzahlWissen}{1}
	\g@addto@macro\TalentTabelleInhaltWissen{#1 (#2 $\bullet$ #3 $\bullet$ #4) & #5 &\\\hline}
}
%
% Handwerkliches Talent
\newcounter{TalentAnzahlHandwerk}
\def\TalentTabelleInhaltHandwerk{}
% command => \neuesTalentHandwerk{Name}{Eig.1}{Eig.2}{Eig.3}{TaW}
\newcommand{\neuesTalentHandwerk}[5]{%
	\addtocounter{TalentAnzahlHandwerk}{1}
	\g@addto@macro\TalentTabelleInhaltHandwerk{#1 (#2 $\bullet$ #3 $\bullet$ #4) & #5 &\\\hline}
}
%
% Sprachen Talent
\newcounter{TalentAnzahlSprachen}
\def\TalentTabelleInhaltSprachen{}
% command => \neuesTalentSprachen{Name}{Kompl.}{TaW}
\newcommand{\neuesTalentSprachen}[3]{%
	\addtocounter{TalentAnzahlSprachen}{1}
	\g@addto@macro\TalentTabelleInhaltSprachen{#1 & #2 & #3 &\\\hline}
}

%--------------------------------------------------------------------
%	Zauberzeug
\newcounter{ZauberAnzahl}
\def\ZauberTabelleInhaltA{}
\def\ZauberTabelleInhaltB{}

% neuen Zauber hinzufügen
% Argumente sind:
%	 1: Zaubername
%	 2: Probe Eig. 1
%	 3: Probe Eig. 2
%	 4: Probe Eig. 3
%	 5: ZfW
%	 6: ZD
%	 7: Kosten
%	 8: Reichweite
%	 9: WD
%	10: Anmerkungen
%	11: Komplexität
%	12: Repräsentation
%	13: Hauszauber?
%	14: Lern
\newcommand{\neuerZauber}[9]{%
\addtocounter{ZauberAnzahl}{1}
\ifnum\value{ZauberAnzahl}>32\g@addto@macro\ZauberTabelleInhaltB{ & #1 & #2 $\bullet$ #3 $\bullet$ #4 & #5 & & #6 & #7 & #8 & #9 &}
\else\g@addto@macro\ZauberTabelleInhaltA{ & #1 & #2 $\bullet$ #3 $\bullet$ #4 & #5 & & #6 & #7 & #8 & #9 &}
\fi
\neuerZauberCont
}
\newcommand{\neuerZauberCont}[5]{%
\ifnum\value{ZauberAnzahl}>32\g@addto@macro\ZauberTabelleInhaltB{ #1 & #2 & #3 & #4 & #5\\\hline}
\else\g@addto@macro\ZauberTabelleInhaltA{ #1 & #2 & #3 & #4 & #5\\\hline}
\fi
}

\makeatother
